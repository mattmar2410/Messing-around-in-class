The energy calibration was performed using a two-point linear fit between
$^{137}$Cs and $^{241}$Am. To perform the calibration, a program searched
the raw spectrum data of $^{137}$Cs and $^{241}$Am looking for
the largest peaks within the spectrum. The program iterated over the spectrum for the
number of gamma-ray energies present since there should only be peaks corresponding
to the number of gamma-ray energies present. Before the program iterated over the raw spectrum,
the data needed to be "cleaned". Noise from the detector could obscure some of the peaks
found from iterating peak heights. With a peak found, the centroid of the peak was
recorded, and subsequently, utilizing a pre-defined width that incapsulates
the whole peak, the program fit a Gaussian and a linear model to this portion of the data.

After modeling the data with the Gaussian and linear model, the peak was
set to zero so during the next iteration the same peak is not found again.
The width of the peak was determined from first analyzing the spectrum and establishing
the average width of each peak. 

Once all of the peaks were discovered, polyfit within python was used to
plot a linear line. The inputs for polyfit were the position of the peak and
the actual gamma-ray energies. The slope-intercept from polyfit was applied
to the channel numbers within the $^{133}$Ba spectrum. Finally, to plot the data
I did the newly calibrated channel numbers vs the original $^{133}$Ba spectrum.
