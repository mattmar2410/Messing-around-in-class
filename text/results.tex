The raw spectrum of $^{133}$Ba is depicted in Figure \ref{fig:raw_data}.

\begin{figure}[H]
\centering
\includegraphics[scale=0.8]{images/Raw_spectrum.png}
\caption{Raw data of $^{133}$Ba produced from a HPGe detector.}
\label{fig:raw_data}
\end{figure}

Inspection of Figure \ref{fig:raw_data} shows that the data has not been calibrated yet.
$^{133}$Ba has 6 gamma-ray energies, but for this analysis, I used five peaks from $^{133}$Ba
detailed in \ref{table:energy} \cite{Untitled27:online}.

\begin{table}[H]
\begin{center}
\begin{tabular}{l|c|c|r}
\textbf{Source} & \textbf{Energy (keV)}\\
\hline
$^{133}$Ba    &  53.1622 \\
              &  80.9979 \\
              &  276.3989 \\
              & 302.8508  \\
              & 356.0129 \\
              & 383.8485 \\
\end{tabular}
\end{center}
\label{table:energy}
\end{table}

I excluded 79.6142 keV from the energy list because it blurs together with
80.99 keV into one photopeak due to the energy resolution
of the HPGe. A better resolution detector would be needed to distinguish
these two peaks. For this reason, I removed it so that way the iterator in the program will
not search for a peak that is not present.

After performing the linear calibration with $^{137}$Cs and $^{241}$Am,
the python function polyfit found the slope and intercept between
the two energies. The equation is shown below.

\begin{equation}
E = 0.28057*x + 1.1810916123
\end{equation}

The slope and intercept was applied to the channel number of the raw Ba133 data.
Figure \ref{fig:CE} depicts the two-point calibration.

\begin{figure}[H]
\centering
\includegraphics[scale=0.8]{images/Ba133_calibrated.png}
\caption{Calibrated $^{133}$Ba data with their corresponding gamma-ray energies
depicted by dashed blue lines}
\label{fig:CE}
\end{figure}
