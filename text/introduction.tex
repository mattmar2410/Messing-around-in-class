In gamma-ray spectroscopy, determining which gamma-rays are present is an important aspect of
radiation detection, and relying solely on the raw channel output makes it difficult to determine the gamma-ray energies
present. By performing an energy calibration, the identification and discernment of
of gamma-ray photopeaks from noise or spurious results becomes more effective. identification
of gamma-ray energies has important functions for non-proliferation reasons. Thus, having
a properly calibrated spectrum is an important component of gamma-ray spectroscopy.
The purpose of Lab 0 was to perform a two-point linear calibration between two gamma-ray photopeaks. The following report
details the process and results of a two-point linear calibration using $^{137}$Cs and $^{241}$Am as
calibration sources.
